\documentclass[10pt,a4paper\arg,uplatex,a4j,dvipdfmx]{jsarticle}
\usepackage[top=25truemm,bottom=25truemm,left=25truemm,right=25truemm]{geometry}
\usepackage[jis2004]{otf}
\usepackage[dvipdfmx]{graphicx}
\usepackage[dvipdfmx]{color}
\usepackage{longtable}
\usepackage{xcolor}

\usepackage{url}
\usepackage{makeidx}
\usepackage{pifont}
\usepackage{here}
\usepackage{multirow}
\usepackage{xparse}
\usepackage{xcolor}
\usepackage{listings}
\usepackage{plext}
\usepackage{amsmath}
\usepackage{amssymb}

\usepackage{multicol}


\usepackage[backend=biber,style=ieee]{biblatex}
\bibliography{report.bib}

% ソースコード
\lstset{
  basicstyle={\ttfamily},
  identifierstyle={\small},
  commentstyle={\smallitshape},
  keywordstyle={\small\bfseries},
  ndkeywordstyle={\small},
  stringstyle={\small\ttfamily},
  frame={tb},
  breaklines=true,
  columns=[l]{fullflexible},
  numbers=left,
  xrightmargin=0zw,
  xleftmargin=3zw,
  numberstyle={\scriptsize},
  stepnumber=1,
  numbersep=1zw,
  lineskip=-0.5ex
}

\usepackage{url}


\newcommand{\cmark}{\ding{51}}%
\newcommand{\xmark}{\ding{55}}%
\newcommand{\ctext}[1]{\hbox{\textcircled{\scriptsize{#1}}}}

\NewDocumentCommand{\codeword}{v}{%
    \texttt{\textcolor{black}{#1}}%
}

\renewcommand{\lstlistingname}{ソースコード}


\begin{document}

    \section*{概要}


    \newpage

    \begin{multicols}{2}

    \section{目的}
    私は、SNS中毒者である\footnote{河井ら \cite{kawai2011sns} のSNS依存尺度に基づく.}.
    その症状として, Twitter\footnote{米Twitter社が運営する, 短文を中心としたSNSサービス. \url{https://twitter.com}}を常に見ていないと落ち着かない\footnote{離脱症状\cite{griffiths2005}}・Twitterを開き, 閉じようと思って閉じても気づくとTwitterを開き直している・Twitterのクライアントアプリを端末から削除しても, ブラウザからアクセスしてしまう\footnote{再燃現象\cite{griffiths2005}}・常にTweetDeckが画面に表示されており, 作業に集中できない・Twitterをなかなか辞められないことに対する罪悪感を感じているも, 結局やめることができない\footnote{葛藤(Intrapersonal conflict)\cite{griffiths2005}}といったものがある.
    
    また, 世の中にはネット中毒\footnote{定義は確立していないようである}と思われる人が多数存在し、(症例を調査する)に悩まされている。
    調査では、
    
    一般人口を代表する参加者:
      ノルウェー: 1.0\%
      米国: 0.7\%
    ヨーロッパ11カ国の学校: 4.4\%
    台湾: ネット中毒の有病率が10.6\%に達したとする調査結果もある\cite{wu2015risk}.
    日本の中高生: 男性6.2\%, 女性9.8\%
   
    \section{概要}
    
    特に携帯電話からのアクセスの場合、依存者ほど利用頻度が高いとする調査結果に注目
    
   ギャンブルと同じような感じで、ドーパミンが… → 断ち切ればいい
   
   流動現象を防ぐために, モニタリングが効果的\cite{internetaddiction}
   モニタリング後の外的ストップテクニックとして使える\cite{internetaddiction}

    \section{動作原理}

    \section{仕様}

    \section{結果}

    \section{考察}

    \end{multicols}


  \printbibliography[title=参考文献]
\end{document}
